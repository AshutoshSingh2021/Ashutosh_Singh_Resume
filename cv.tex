%-----------------------------------------------------------------------------------------------------------------------------------------------%
%	The MIT License (MIT)
%
%	Copyright (c) 2021 Jitin Nair
%
%	Permission is hereby granted, free of charge, to any person obtaining a copy
%	of this software and associated documentation files (the "Software"), to deal
%	in the Software without restriction, including without limitation the rights
%	to use, copy, modify, merge, publish, distribute, sublicense, and/or sell
%	copies of the Software, and to permit persons to whom the Software is
%	furnished to do so, subject to the following conditions:
%	
%	THE SOFTWARE IS PROVIDED "AS IS", WITHOUT WARRANTY OF ANY KIND, EXPRESS OR
%	IMPLIED, INCLUDING BUT NOT LIMITED TO THE WARRANTIES OF MERCHANTABILITY,
%	FITNESS FOR A PARTICULAR PURPOSE AND NONINFRINGEMENT. IN NO EVENT SHALL THE
%	AUTHORS OR COPYRIGHT HOLDERS BE LIABLE FOR ANY CLAIM, DAMAGES OR OTHER
%	LIABILITY, WHETHER IN AN ACTION OF CONTRACT, TORT OR OTHERWISE, ARISING FROM,
%	OUT OF OR IN CONNECTION WITH THE SOFTWARE OR THE USE OR OTHER DEALINGS IN
%	THE SOFTWARE.
%	
%
%-----------------------------------------------------------------------------------------------------------------------------------------------%

%----------------------------------------------------------------------------------------
%	DOCUMENT DEFINITION
%----------------------------------------------------------------------------------------

% article class because we want to fully customize the page and not use a cv template
\documentclass[a4paper,12pt]{article}

%----------------------------------------------------------------------------------------
%	FONT
%----------------------------------------------------------------------------------------

% % fontspec allows you to use TTF/OTF fonts directly
% \usepackage{fontspec}
% \defaultfontfeatures{Ligatures=TeX}

% % modified for ShareLaTeX use
% \setmainfont[
% SmallCapsFont = Fontin-SmallCaps.otf,
% BoldFont = Fontin-Bold.otf,
% ItalicFont = Fontin-Italic.otf
% ]
% {Fontin.otf}

%----------------------------------------------------------------------------------------
%	PACKAGES
%----------------------------------------------------------------------------------------
\usepackage{url}
\usepackage{parskip} 	

%other packages for formatting
\RequirePackage{color}
\RequirePackage{graphicx}
\usepackage[usenames,dvipsnames]{xcolor}
\usepackage[scale=0.9]{geometry}

%tabularx environment
\usepackage{tabularx}

%for lists within experience section
\usepackage{enumitem}

% centered version of 'X' col. type
\newcolumntype{C}{>{\centering\arraybackslash}X} 

%to prevent spillover of tabular into next pages
\usepackage{supertabular}
\usepackage{tabularx}
\newlength{\fullcollw}
\setlength{\fullcollw}{0.47\textwidth}

%custom \section
\usepackage{titlesec}				
\usepackage{multicol}
\usepackage{multirow}

%CV Sections inspired by: 
%http://stefano.italians.nl/archives/26
\titleformat{\section}{\Large\scshape\raggedright}{}{0em}{}[\titlerule]
\titlespacing{\section}{0pt}{7pt}{7pt}

%for publications
\usepackage[style=authoryear,sorting=ynt, maxbibnames=2]{biblatex}

%Setup hyperref package, and colours for links
\usepackage[unicode, draft=false]{hyperref}
\definecolor{linkcolour}{rgb}{0,0.2,0.6}
\hypersetup{colorlinks,breaklinks,urlcolor=linkcolour,linkcolor=linkcolour}
\addbibresource{citations.bib}
\setlength\bibitemsep{1em}

%for social icons
\usepackage{fontawesome5}

%debug page outer frames
%\usepackage{showframe}

%----------------------------------------------------------------------------------------
%	BEGIN DOCUMENT
%----------------------------------------------------------------------------------------
\begin{document}

% non-numbered pages
\pagestyle{empty} 

%----------------------------------------------------------------------------------------
%	TITLE
%----------------------------------------------------------------------------------------

% \begin{tabularx}{\linewidth}{ @{}X X@{} }
% \huge{Your Name}\vspace{2pt} & \hfill \emoji{incoming-envelope} email@email.com \\
% \raisebox{-0.05\height}\faGithub\ username \ | \
% \raisebox{-0.00\height}\faLinkedin\ username \ | \ \raisebox{-0.05\height}\faGlobe \ mysite.com  & \hfill \emoji{calling} number
% \end{tabularx}

\begin{tabularx}{\linewidth}{@{} C @{}}
\Huge{Ashutosh Singh} \\[4.0pt]
\href{https://github.com/AshutoshSingh2021}{\raisebox{-0.05\height}\faGithub\ AshutoshSingh2021} \ $|$ \ 
\href{https://www.linkedin.com/in/ashutosh-18-in/}{\raisebox{-0.05\height}\faLinkedin\ ashutosh-18-in} \ $|$ \ 
\href{mailto:ashutoshsingh15102@gmail.com}{\raisebox{-0.05\height}\faEnvelope \ ashulearner18@gmail.com} \ $|$ \ 
\href{tel:+916387724476}{\raisebox{-0.05\height}\faMobile \ +91 6387724476} \\
\end{tabularx}

%----------------------------------------------------------------------------------------
% EXPERIENCE SECTIONS
%----------------------------------------------------------------------------------------

%Experience
\section{Work Experience}

\begin{tabularx}{\linewidth}{ @{}l r@{} }
\textbf{Open-Source contributor, JWOC} & \hfill \textbf{Feb 2023 - March 2023} \\[2.00pt]
\multicolumn{2}{@{}X@{}}{
\begin{minipage}[t]{\linewidth}
    \begin{itemize}[nosep,after=\strut, leftmargin=1em, itemsep=1.00pt]
        \item[--]  Worked majorly on a \href{https://seamlessui.vercel.app/}{\underline{Tailwind CSS library}}, created different "tailwind components". I used "Git" and "GitHub" to collaborate and submit my work. I improved my work by seeking constant feedback from the maintainers, and 7 pull requests were merged.
        \item[--] I worked on another repository \href{https://github.com/Srijita-Mandal/fix-your-nums}{\underline{fix-your-nums}} which is a calculator application. I have resolved a "front-end" bug that was causing the table to not render properly. I used "HTML", "CSS" and "JavaScript" to resolve this issue.
    \end{itemize}
    \end{minipage}
}
\end{tabularx}

%Projects
\section{Projects}

\begin{tabularx}{\linewidth}{ @{}l r@{} }
\textbf{Medvisor} & \hfill \href{https://github.com/AshutoshSingh2021/medvisor}{Project Link} \\[2.00pt]
\multicolumn{2}{@{}X@{}}{\begin{minipage}[t]{\linewidth}
    \begin{itemize}[nosep,after=\strut, leftmargin=1em, itemsep=1.00pt]
        \item[--] Medvisor is an AI fitness advisor. It can answer most of the queries about fitness and health.
        \item[--] Worked mainly on the "backend" part. Utilised "Node.js" as a runtime environment, "Express.js" for creating the REST API and "MongoDB" as a NoSQL database.
        \item[--] Harnessed the Gemini API with the "gemini-pro" model to handle user prompts. Apart from that, I also used "bcryptjs" for password encryption and the "jsonwebtoken" library for user authentication.
        \item[--] Also connected the frontend with the backend and set up CORS to handle cross-origin requests. Implemented the functionality to render the raw AI responses in the formatted text on the front-end.
        \item[--] \textbf{Tech used: }React, Node.js, Express.js, JavaScript, RestAPIs, 
 , NoSQL, MERN stack, Postman
    \end{itemize}
    \end{minipage}}  \\
\end{tabularx}

\begin{tabularx}{\linewidth}{ @{}l r@{} }
\textbf{Ticket Resolution App} & \hfill \href{https://github.com/AshutoshSingh2021/Ticket-resolution-app}{Project Link} \\[2.00pt]
\multicolumn{2}{@{}X@{}}{\begin{minipage}[t]{\linewidth}
    \begin{itemize}[nosep,after=\strut, leftmargin=1em, itemsep=1.00pt] 
        \item[--] The Objective was to create a Ticket resolution application to resolve doubts and assign tickets to the moderators automatically using AI.
        \item[--] Created a Backend using Node.js, and a basic frontend with React.js.
        \item[--] Utilised Inngest framework to integrate AI for assignment of tickets to the moderators. Currently working to make it better.
        \item[--] \textbf{Tech used: } Node.js, Express.js, React, MongoDB, Gemini, Inngest, Nodemailer, UI component library
    \end{itemize}
    \end{minipage}}  \\
\end{tabularx}

\begin{tabularx}{\linewidth}{ @{}l r@{} }
\textbf{TextUtils} & \hfill \href{https://github.com/AshutoshSingh2021/my-First-react-app}{Project Link} \\[2.00pt]
\multicolumn{2}{@{}X@{}}{\begin{minipage}[t]{\linewidth}
    \begin{itemize}[nosep,after=\strut, leftmargin=1em, itemsep=1.00pt] 
        \item[--] TextUtils is a Text formatting tool that can transform huge text according to your needs.
        \item[--] Utilised "React.js" and "Bootstrap" libraries to create TextUtils. It can convert the text in different cases, remove extra spaces, extract all the emails and more.
        \item[--] \textbf{Tech used: } JavaScript, Bootstrap, React.js
    \end{itemize}
    \end{minipage}}  \\
\end{tabularx}
%----------------------------------------------------------------------------------------
%	EDUCATION
%----------------------------------------------------------------------------------------
\section{Education}
\begin{tabularx}{\linewidth}{@{}l X@{}}	
20\textbf{20} - 20\textbf{24} & \textbf{B.Tech} (CSE) at \textbf{Dr. A. P. J. Abdul Kalam Technical University} \hfill \normalsize GPA: \textbf{7.4/10.0} \\
20\textbf{18} - 20\textbf{19} & \textbf{Intermediate} (Science) \textbf{Saraswati Vidya Mandir} (CBSE) \hfill  \textbf{75\%} \\

\end{tabularx}

%----------------------------------------------------------------------------------------
%	PUBLICATIONS
%----------------------------------------------------------------------------------------
%----------------------------------------------------------------------------------------
%	SKILLS
%----------------------------------------------------------------------------------------
\section{Technical Skills}
\begin{tabularx}{\linewidth}{@{}l X@{}}
\textbf{Languages: } &  \normalsize{JavaScript, C/C++}\\
\textbf{Cloud/Database: }  &  \normalsize{MongoDB}\\
\textbf{Developer Tools: }  &  \normalsize{VS Code, Git, GitHub, Postman}\\
\textbf{Technology/Framework: }  &  \normalsize{React.js, Express.js, Node.js, HTML, CSS, Bootstrap, Tailwind, Vite}\\
\textbf{Others: }  &  \normalsize{}\\
\end{tabularx}

\end{document}
